\anforderung[
    ursprung=Rail:UC:Tfz,
]
    {Rail:Komm:Integr:Tfz}
    {Kommunikationsinfrastruktur}
    {muss}
    {Integrität der Datenverbindung zwischen Stellwerk und Triebfahrzeug}
    {sicherstellen}

\begin{figure}[!htb]
  \centering
  \includegraphics[max width=\textwidth]{abbildungen/SysML1_6_UseCaseDiagram_OBU_IXL.pdf}
  \caption{Anwendungsfalldiagramm (\proper{SysML}) Stellwerk Triebfahrzeug}
\end{figure}

\entscheidung
  {Ent:Beispiel} % Kennung
  {\useid{Rail:Req:DistanzUnlimitiert}} % Problem
  {\usealt{1} Es wird der SuperLaser3000 der Fa.\ ThüriLicht verwendet} % Lösung
  {\defalt{1} Handelsüblicher Laserpointer als Photonenquelle: Günstig aber technisch ungenügend.} % Alternativen
  {vorgeschlagen} % Status

Noch einiges an Text.

\paragraph{Nichtfunktionale Anforderungen}

\begin{itemize}
    \item \anforderung
      {All:Req:quantensicher}
      {Das System}
      {muss}
      {gegenüber kryptographischen Angriffen mit Quantencomputern}
      {die IT-Sicherheit wahren}

    \item \anforderung
      {Rail:Req:Schlüsselrate}
      {Das System}
      {muss}
      {Schlüssel in einer für das erwartete Datenvolumen ausreichenden Rate}
      {bereitstellen}
\end{itemize}
